\chapter{Abstract}
\textit{Fittig} is the process of finding the optimal parameters for a 3D Morphable Model (3DMM) so that its as close as possible to a given 2D-Object. In our our case, these objects are facial images of which only the face should be reconstructed. The Fitting-Algorithm has somehow to determine whether a pixel of the given image belongs to the face or not. One approach is to label each pixel if it shows a part of the face. The process of finding these labels is called \textit{segmentation} and all labels together make up the \textit{mask}. In facial images, faces are often not completely visible or occluded by a variety of objects. Therefore, an appropriate segmentation should detect occlusions and exclude them from the facial region.\\
\\
In this thesis, we use the face segmentation-network proposed by Nirkin et al. \cite{nirkin2018_faceswap}. It is claimed to be very fast and to make accurate segmentations. In a first step, we evaluated the mask of the network by comparison with other segmentations. In a next step, we integrate the mask of the network into the combined segmentation and parameter adaption process of Egger et al. \cite{egger_paper} by treating our segmentation as a mask for the fitting process. We compare these fits with results of other masks by calculating the difference of the parameters.\\

